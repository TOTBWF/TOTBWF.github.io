% Created 2021-02-07 Sun 11:04
% Intended LaTeX compiler: pdflatex
\documentclass[11pt]{article}
\usepackage[utf8]{inputenc}
\usepackage[T1]{fontenc}
\usepackage{graphicx}
\usepackage{grffile}
\usepackage{longtable}
\usepackage{wrapfig}
\usepackage{rotating}
\usepackage[normalem]{ulem}
\usepackage{amsmath}
\usepackage{textcomp}
\usepackage{amssymb}
\usepackage{capt-of}
\usepackage{hyperref}
\author{Reed Mullanix}
\date{\today}
\title{}
\hypersetup{
 pdfauthor={Reed Mullanix},
 pdftitle={},
 pdfkeywords={},
 pdfsubject={},
 pdfcreator={Emacs 28.0.50 (Org mode 9.4)}, 
 pdflang={English}}
\begin{document}

\setcounter{tocdepth}{2}
\tableofcontents


\section{Introduction}
\label{sec:orgfe8c863}
For Eliot, Merry Christmas! This page will be continuously updated, so
check back in every once and a while! A PDF version is also available

\section{Pantry}
\label{sec:org1e01655}
\subsection{Dashi}
\label{sec:org1e7046f}
\subsubsection{Ingredients}
\label{sec:org6d6f8ff}
\begin{itemize}
\item 30 grams dried Kombu
\item 30 grams bonito flakes
\item 2 Liters water
\end{itemize}
\subsubsection{Instructions}
\label{sec:org599bdcf}
Combine water and kombu in a saucepan, and bring to a simmer over
medium heat. Remove from the heat, then add in the bonito flakes,
and let sit for 5 minutes. Strian off the kombu and bonito and
discard. The resulting stock can be stored in the fridge for up to
a week.
\section{Snacks}
\label{sec:org0588869}
\subsection{Sourdough Starter Scallion Pancakes}
\label{sec:orgcf080ec}
\begin{itemize}
\item Leftover sourdough starter
\item Scallions
\item Sesame Oil
\item Salt
\end{itemize}
\subsubsection{Instructions}
\label{sec:org2acc70f}
Dice scallions, and add to leftover starter and mix to form a
batter. Heat a pan on medium heat, and pour batter down to form
pancakes. Cook as you would a pancake.

In a small bowl, mix together a bit of sesame oil and salt.
To serve, dip the pancakes in the sauce.
\section{Lunch}
\label{sec:org8788c91}
\subsection{Chickpea Wraps}
\label{sec:org0c0c69d}
\subsubsection{Ingredients}
\label{sec:orgad44150}
\begin{itemize}
\item 1 can chickpeas (15 oz)
\item Mayonnaise (1/3 cup)
\item Sriracha (1.5 Tbsp)
\item Cilantro (2 Tbsp)
\item Lemon Juice (1 Tsp)
\item Salt (1/8 Tsp)
\item Large Flour Tortillas (2)
\item Spinach (2 Cups)
\item Carrot (1)
\item Bell Pepper (1/2)
\end{itemize}
\subsubsection{Prep}
\label{sec:orgb85426f}
\begin{itemize}
\item Drain chickpeas
\item Thinly dice carrot into strips
\item Cut bell pepper into strips
\end{itemize}
\subsubsection{Instructions}
\label{sec:org85792ea}
Add drained chickpeas then add them to a bowl with the mayonnaise,
sriracha, cilantro, lemon juice, and salt. (You may want to start
with less sriracha and add more to taste if you are sensitive to
heat). Mash the chickpeas until they are fairly broken down and
the dressing ingredients have combined.

To build the wraps, place the tortillas on a work surface and add
half of the spinach, carrot sticks, bell pepper, and sriracha
chickpea salad to each one. Fold the sides in toward the center,
then roll from the bottom up, like a burrito. Serve immediately or
refrigerate until ready to eat.
\section{Dinner}
\label{sec:org8d31b38}
\subsection{Chickpeas and Orzo}
\label{sec:orgd8e01ae}
\subsubsection{Ingredients}
\label{sec:orgf590341}
\begin{itemize}
\item 1 can chickpeas (15oz)
\item Orzo (2 servings)
\item Shallot (2)
\item Italian Parsley (5 sprigs)
\item Feta Cheese
\item Olive Oil
\end{itemize}
\subsubsection{Prep}
\label{sec:org89fb210}
\begin{itemize}
\item Dice Parsley
\item Crumble feta cheese
\item Drain and rinse chickpeas
\end{itemize}
\subsubsection{Instructions}
\label{sec:orgf1b5a5b}
Boil water, salt, and cook orzo.

Saute the shallots until they are translucent and begin
to brown. Then, add the chickpeas and stir, until they are warmed.
Then, add the orzo and mix. Remove from heat, and add feta cheese
to taste. Sprinkle with diced parsley.
\subsection{Korean Rice Bowls}
\label{sec:org86f9c2a}
\subsubsection{Ingredients}
\label{sec:org0bf96cd}
\begin{itemize}
\item 2 Cups Cooked Rice
\item 2 Tbsp Gochujang
\item 4 Tbsp Soy Sauce
\item 1 Tbsp Rice Vinegar
\item 2 Tbsp Water
\item 4 Large Portobello Mushrooms
\item 1 Yellow Onion
\item 1 Cucumber
\item 1 Carrot
\item 2 green onions
\item Sesame Seeds
\end{itemize}
\subsubsection{Prep}
\label{sec:org59aa65f}
\begin{itemize}
\item Slice Onion
\item Grate Carrot
\item Slice Cucumber on bias
\item Slice Green Onions on bias
\item Slice Mushrooms in half, then crosswise.
\end{itemize}
\subsubsection{Instructions}
\label{sec:orga21e61b}
First, begin by preparing the Mushroom Marinade. Mix together the
Gochujang, Soy Sauce, Rice Vinegar, and Water. Add mushrooms to marinade.
Mix till all the mushrooms are coated, and let stand for 10 minutes.

Mix together 1Tbsp Rice vinegar, 1 Tsp sugar, and a sprinkle of
salt. Add cucumber and let stand, occasionally mixing.

Heat pan with a smal amount of oil on medium-high. Add onions and let cook
until translucent. Add mushrooms and marinade then cook till sauce
is somewhat reduced, about 7-9 minutes.

Prepare rice bowls by laying down rice, adding mushroom/onions,
then cucumbers, then carrots. Sprinkle with green onion and sesame seeds.

\subsection{Sesame Noodles}
\label{sec:orgd71dade}
\subsubsection{Ingredients}
\label{sec:orgb0e87d7}
\begin{itemize}
\item 4 scallions, whites and greens separated, thinly sliced
\item 1/2 cup vegetable oil
\item 1 tablespoon crushed red pepper flakes
\item 2 teaspoons sesame seeds
\item 2 teaspoons Sichuan pepper, coarsely chopped
\item 12 ounces thin noodles
\item Kosher salt
\item 1/4 cup tahini (sesame seed paste)
\item 1/4 cup unseasoned rice vinegar
\item 3 tablespoons soy sauce
\item 2 teaspoons toasted sesame oil
\item 1 teaspoon sugar
\end{itemize}
\subsubsection{Instructions}
\label{sec:org5522268}
Cook scallion whites, vegetable oil, red pepper flakes, sesame
seeds, and pepper in a small saucepan over low heat, stirring
occasionally, until oil is sizzling and scallions are golden
brown, 12–15 minutes; let chili oil cool in saucepan.

Meanwhile, cook noodles in a large pot of salted boiling water
until al dente; drain. Rinse under cold water and drain well.

Whisk tahini, vinegar, soy sauce, sesame oil, sugar, and 2–3
tablespoons chili oil (depending on desired heat) in a large bowl;
season with salt. Add noodles and toss to coat. Top with scallion
greens and drizzle with more chili oil.

\subsection{Chicken Paprikish}
\label{sec:orgc20ee7d}
\subsubsection{Ingredients}
\label{sec:orgfcc972c}
\begin{itemize}
\item 1 Yellow Onion
\item 1 Red Pepper
\item 1/4 Cup Hungarian Sweet Paprika
\item 1 Bay Leaf
\item 1 Cup Chicken Stock
\item 4 Chicken Legs
\item 1/2 Sour Cream
\item Dill
\end{itemize}
\subsubsection{Prep}
\label{sec:orgb72bbf3}
Slice onions and bell peppers into medium sized slivers. Split
chicken legs into thighs and drumsticks.
\subsubsection{Instructions}
\label{sec:org3d1ff3d}
Season chicken on all sides with salt and pepper. Heat a small
amount of high-smokepoint oil in a pan over medium heat. Once the
oil is hot, add the chicken pieces skin side down and cook without
disturbing for about 8 minutes. Flip over and cook until the other
side for about 2 minutes. Move chicken to a plate, and remove all
but a small amount of fat from the pan.

Saute onions and red peppers on medium heat for about 7 minutes.
Add the paprika and cook while stirring rapidly, making sure not
to let the paprika burn, for 1 minute. Add the stock, and scrape
up anything stuck to the bottom of the pan. Add the bay leaf, then
place the chicken in a single layer into the pan, skin side up.

Place the heat on low, and cover the pan and cook until the
chicken is tender all the way through, which ought to take about
30 minutes.

Once the chicken is cooked, remove it, and add the sour cream into
the sauce. Stir until it is mixed thoroughly, then readd the
chicken back in the sauce, turning to coat.

When serving, garnish with extra sour cream and minced dill.
\subsection{Oyakodon}
\label{sec:org5b58547}
\subsubsection{Ingredients}
\label{sec:orgb67a099}
\begin{itemize}
\item 1 Cup \hyperref[sec:org1e7046f]{Dashi}
\item 1 Tbsp Soy Sauce
\item 2 Tbsp Sake
\item 1 Tbsp Sugar
\item 1 Large Onion
\item 340 grams Chicken Thighs
\item 3 Green Onions
\item 4 Eggs
\end{itemize}
\subsubsection{Prep}
\label{sec:org1795e81}
Slice onions into slivers. Slice Scalions, and divide into 2 piles.
Thinly slice chicken
\subsubsection{Instructions}
\label{sec:org7d7e3fb}
Combine Dashi, Soy Sauce, Sake, and sugar in a saucepan and bring
to a simmer over high heat. Stir in onion and cook until the
onion is just barely tender, about 5 minutes. Add chicken slices
and cook, stirring occasionally. Continue until chichken is cooked
through, about 6 minutes. Stir in half of the green onions, and
season the broth with soy sauce and sugar to taste.

Reduce heat, maintaining a bare simmer. Beat eggs in a medium
bowl, and slowly pour the eggs into the pot in a thin, steady
stream, using chopsticks to control the flow. Cover and cook until
the eggs are cooked to your liking, which should be in the 1-3
minute range.

To serve, place cooked rice in bowls, and top with egg and chicken
mixture, adding any extra broth to the rice. Garnish with the
remaining green onions.
\section{Baking}
\label{sec:orgf8b2e15}
\subsection{Overnight White Bread}
\label{sec:orgd8daee8}
\begin{itemize}
\item White Flour (500g)
\item Kosher Salt (14 g)
\item Yeast (1/4 Tsp)
\end{itemize}
\subsubsection{Instructions}
\label{sec:orga19094e}
Mix together flour and 390 grams of warm water in a bowl until no
dry flour remains, and let stand covered with a cloth for 30 minutes.

Next, add the salt and yeast and fold until incorporated. Then,
gently fold the dough in half 4 or 5 times, or until the dough
becomes harder to work. Let rest for 30 minutes, and repeat this
folding process 3 more times, waiting 30 minutes between each.

Once the folds are complete, let the dough sit overnight covered
by a towel (roughly 8 to 12 hours).

Once this bulk rise has completed, turn the dough onto an
\uline{unfloured} work surface. Working quickly, grab the right hand
side of the bread, and fold 2/3ds of the way over. Then, grab the
left hand side, and do the same. Next, grab the back side of the
bread and fold over as well. Now, roll the dough towards you until
the seam is on the bottom. Now, take the dough in both hands and
rotate it around to form a ball, making sure not to lift it from
the work surface. Once the outer skin of the dough has built up
sufficient tension, dust the top with flour until it is no longer
tacky, and place in a bowl lined with a towel, floured side down.

Let the dough rest for about 45 minutes. In the mean time, place a
dutch oven in the oven at 475 degrees, and let it heat up. When
the resting period is done, quickly turn out the dough onto a
piece of parchment paper. Score the top using a sharp knife, and
then lift the parchment paper into the hot dutch oven. Cover and
let bake for 20 minutes. Once the 20 minutes is up, uncover and
let bake until the top is a rich brown. Remove, and let cool.
\subsection{Sourdough Starter}
\label{sec:org3e6f48c}
To start, mix together a 50/50 blend of whole wheat and white
flour. Then add warm water slowly and mix, until you reach a batter
like consistency. Leave in a warm, dark place covered with a paper towel
for 3 days, checking every day. Around the third day, the starter
should start to smell a bit funky and fruity, and should have grown
in size or formed bubbles.

Once you have cultivated your starter, it is quite easy to use,
feed, and maintain it. To use, discard all but one tablespoon of
the starter, and feed again with a 50/50 blend of whole wheat and
white flour and water, getting it to a batter like consistency.
Let this develop for 24 hours. You will know it is ready when it
floats when placed in water.

For the overnight white loaf, I like to use about 100 grams of
starter, and reduce the water to about 370 grams.

Starter is easiest to work with when you get into a rythm, as
discarded starter will be used in a loaf of bread, and you can feed
at the same time as making the dough to be ready for the next day.
However, baking a loaf of bread every day is a big ask, and not
always realistic! In these cases, you can put the starter in a
sealed jar in the fridge to make it go dormant. Just make sure to
check on it every once and a while to make sure it doesn't dry out.

\subsection{Japanese Milk Bread}
\label{sec:org2efd08a}
\subsubsection{Ingredients}
\label{sec:org83cc391}
\begin{enumerate}
\item Roux
\label{sec:org14502da}
\begin{itemize}
\item 20 grams flour
\item 100 grams water
\end{itemize}
\item Dough
\label{sec:org3d9225a}
\begin{itemize}
\item 330 grams flour
\item 24 grams sugar
\item 14 grams powdered milk
\item 7 grams salt
\item 4 grams instant yeast
\item 95 grams warm water
\item 1 egg
\item 20 grams butter
\end{itemize}
\end{enumerate}
\subsubsection{Instructions}
\label{sec:org72b8a59}
First, mix together all the dry ingredients from the \textbf{Dough}
section. Then, proceed to make the roux by whisking the flour into
the water in a small pot. Heat on medium low, stirring
continuously. It is done when pulling a spatula through it allows
you to see the bottom.

Add all the water, egg, and roux to the dry ingredient
mixture. Mix, then knead until the dough comes together. Then,
slowly incorporate the butter while kneading, until the dough is
slightly tacky. Shape the dough into a ball, then place in a
greased bowl.

Cover the dough, and let rise for about an hour, or until it has
roughly doubled in size. Punch down the dough, divide into halves,
and reshape into balls. Let the dough rest again for 20 minutes on
the counter.

Roll out each ball into an elongated elipse, and fold into thirds lengthwise.
Roll out the dough again in the same manner, then roll the dough
up into a log.

Place both rolls in a loaf pan with the seam facing down. Let
the dough proof, until it has filled up the pan. Bake at 350F for
35 minutes. Remove from the pan and let cool on a rack.
\end{document}
